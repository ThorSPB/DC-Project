\chapter{State of the art}

\section{Cinebench}

\begin{itemize}
  \item Cinebench is a free benchmarking tool developed by Maxon, designed to evaluate your computer's CPU and GPU performance by rendering complex 3D scenes. It simulates real-world tasks to assess how well your hardware handles intensive workloads, making it useful for anybody stress-testing their desktop.
  \item Cinebench provides separate tests for single-core and multi-core CPU performance, as well as GPU rendering capabilities, offering a comprehensive view of your system's processing power. It's widely used by hardware reviewers, system builders, and enthusiasts to compare performance across different systems and configurations.
\end{itemize}

\section{UserBenchmark}

\begin{itemize}
  \item UserBenchmark is a free benchmarking tool and website that allows users to evaluate and compare the performance of their computer hardware, including CPUs, GPUs, SSDs, HDDs, RAM, and USB drives.
  \item UserBenchmark displays the results in percentages. For example, if a CPU has scored 120\% in the test, that means it completed the test 20\% "average" CPU.
  \item It should be noted that the CPU test in UserBenchmark faced some backlash for placing too much importance in single-core performance.
\end{itemize}

\section{Comparison}
\begin{table}[ht]
\centering
\begin{tabular}{|l|l|l|l|}
\hline
                      & \textbf{Cinebench}  & \textbf{UserBenchmark} & \textbf{HappyBench}  \\ \hline
Single-core only test & Yes                 & No                     & No                      \\ \hline
Hardware Tested       & CPU, GPU            & CPU, GPU, RAM, SSD, USB& CPU, SSD             \\ \hline
Results format        & Score               & Percentage             & Time, Percentage \\
\hline
\end{tabular}
\label{example-table}
\end{table}


